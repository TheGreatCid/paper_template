%%%Author definitions
\def\tsc#1{\csdef{#1}{\textsc{\lowercase{#1}}\xspace}}
\tsc{WGM}
\tsc{QE}
\tsc{EP}
\tsc{PMS}
\tsc{BEC}
\tsc{DE}
%%%

% Uncomment and use as if needed
%\newtheorem{theorem}{Theorem}
%\newtheorem{lemma}[theorem]{Lemma}
%\newdefinition{rmk}{Remark}
%\newproof{pf}{Proof}
%\newproof{pot}{Proof of Theorem \ref{thm}}

\let\WriteBookmarks\relax
\def\floatpagepagefraction{1}
\def\textpagefraction{.001}

% Short title
\shorttitle{Leveraging social media news}

% Short author
\shortauthors{CV Radhakrishnan et~al.}

% Main title of the paper
\title [mode = title]{This is a specimen $a_b$ title}                      
% Title footnote mark
% eg: \tnotemark[1]
\tnotemark[1,2]

% Title footnote 1.
% eg: \tnotetext[1]{Title footnote text}
% \tnotetext[<tnote number>]{<tnote text>} 
\tnotetext[1]{This document is the results of the research
   project funded by the National Science Foundation.}

\tnotetext[2]{The second title footnote which is a longer text matter
   to fill through the whole text width and overflow into
   another line in the footnotes area of the first page.}


% First author
%
% Options: Use if required
% eg: \author[1,3]{Author Name}[type=editor,
%       style=chinese,
%       auid=000,
%       bioid=1,
%       prefix=Sir,
%       orcid=0000-0000-0000-0000,
%       facebook=<facebook id>,
%       twitter=<twitter id>,
%       linkedin=<linkedin id>,
%       gplus=<gplus id>]
\author[1,3]{CV Radhakrishnan}[type=editor,
                        auid=000,bioid=1,
                        prefix=Sir,
                        role=Researcher,
                        orcid=0000-0001-7511-2910]

% Corresponding author indication
\cormark[1]

% Footnote of the first author
\fnmark[1]

% Email id of the first author
\ead{cvr_1@tug.org.in}

% URL of the first author
\ead[url]{www.cvr.cc, cvr@sayahna.org}

%  Credit authorship
% \credit{Conceptualization of this study, Methodology, Software}

% Address/affiliation
\affiliation[1]{organization={Elsevier B.V.},
    addressline={Radarweg 29}, 
    city={Amsterdam},
    % citysep={}, % Uncomment if no comma needed between city and postcode
    postcode={1043 NX}, 
    % state={},
    country={The Netherlands}}

% Second author
\author[2,4]{Han Theh Thanh}[style=chinese]

% Third author
\author[2,3]{CV Rajagopal}[%
   role=Co-ordinator,
   suffix=Jr,
   ]
\fnmark[2]
\ead{cvr3@sayahna.org}
\ead[URL]{www.sayahna.org}

% \credit{Data curation, Writing - Original draft preparation}

% Address/affiliation
\affiliation[2]{organization={Sayahna Foundation},
    % addressline={}, 
    city={Jagathy},
    % citysep={}, % Uncomment if no comma needed between city and postcode
    postcode={695014}, 
    state={Trivandrum},
    country={India}}

% Fourth author
\author%
[1,3]
{Rishi T.}
\cormark[2]
\fnmark[1,3]
\ead{rishi@stmdocs.in}
\ead[URL]{www.stmdocs.in}

\affiliation[3]{organization={STM Document Engineering Pvt Ltd.},
    addressline={Mepukada}, 
    city={Malayinkil},
    % citysep={}, % Uncomment if no comma needed between city and postcode
    postcode={695571}, 
    state={Trivandrum},
    country={India}}

% Corresponding author text
\cortext[cor1]{Corresponding author}
\cortext[cor2]{Principal corresponding author}

% Footnote text
\fntext[fn1]{This is the first author footnote. but is common to third
  author as well.}
\fntext[fn2]{Another author footnote, this is a very long footnote and
  it should be a really long footnote. But this footnote is not yet
  sufficiently long enough to make two lines of footnote text.}

% For a title note without a number/mark
\nonumnote{This note has no numbers. In this work we demonstrate $a_b$
  the formation Y\_1 of a new type of polariton on the interface
  between a cuprous oxide slab and a polystyrene micro-sphere placed
  on the slab.
  }

% Here goes the abstract
\begin{abstract}
This template helps you to create a properly formatted \LaTeX\ manuscript.

\noindent\texttt{\textbackslash begin{abstract}} \dots 
\texttt{\textbackslash end{abstract}} and
\verb+\begin{keyword}+ \verb+...+ \verb+\end{keyword}+ 
which
contain the abstract and keywords respectively. 

\noindent Each keyword shall be separated by a \verb+\sep+ command.
\end{abstract}

% Use if graphical abstract is present
% \begin{graphicalabstract}
% \includegraphics{figs/grabs.pdf}
% \end{graphicalabstract}

% Research highlights
% \begin{highlights}
% \item Research highlights item 1
% \item Research highlights item 2
% \item Research highlights item 3
% \end{highlights}

% Keywords
% Each keyword is seperated by \sep
\begin{keywords}
quadrupole exciton \sep polariton \sep \WGM \sep \BEC
\end{keywords}


\maketitle

\printcredits